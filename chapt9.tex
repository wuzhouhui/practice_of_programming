% vim: ts=4 sts=4 sw=4 et tw=75
\chapter{Notation}
\label{chap:notation}
\begin{quote}
    Perhaps of all the creations of man language is the most astonishing
    (惊讶的).
\end{quote}
\begin{quotesrc}
    Giles Lytton Strachey, \bookname{Words and Poetry}
\end{quotesrc}

The right language can make all the difference in how easy it is to write a
program. This is why a practicing programmer's arsenal holds not only
general-purpose languages like C and its relatives, but also programmable
shells, scripting languages, and lots of application-specific languages

The power of good notation reaches beyond traditional programming into
specialized problem domains. Regular expressions let us write compact (if
occasionally cryptic (神秘的)) definitions of classes of strings; HTML lets
us define the layout of interactive documents, often using embedded
programs in other languages such as JavaScript; Postscript expresses an
entire document -- this book, for example -- as a stylized program.
Spreadsheets and word processors often include programming languages like
Visual Basic to evaluate expressions, access information, or control
layout.

If you find yourself writing too much code to do a mundane (平凡的) job, or
if you have trouble expressing the process comfortably, maybe you're using
the wrong language. If the right language doesn't yet exist, that might be
an opportunity to create it yourself. Inventing a language doesn't
necessarily mean building the successor to Java; often a thorny (多刺的)
problem can be cleared up by a change of notation. Consider the format
strings in the \verb'printf' family, which are a compact and expressive way
to control the display of printed values.

In this chapter, we'll talk about how notation can solve problems, and
demonstrate some of the techniques you can use to implement your own
special-purpose languages. We'll even explore the possibilities of having
one program write another program, an apparently extreme use of notation
that happens more often, and is far easier to do, than many programmers
realize.
